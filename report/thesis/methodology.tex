\chapter{Methodology}

Now that we have looked at the background and previous work in the Clinical Decision Support area,
let us proceed to formally introducing our own retrieval system. 

Recall that the medical queries we focus on have two components:
\begin{enumerate*}[label=\arabic*)]
 \item the \emph{query text}, containing the medical case report (patient history, current symptoms, any test results) and
 \item the \emph{intent type}, that is, the generic clinical question the physician is interested in (diagnosis, test, or treatment).
\end{enumerate*}
Therefore, we formally represent medical queries $Q$ as pairs of text and intent type: \[Q = \{Q.text,\ Q.type\}\]

Given a query $Q$, the retrieval system computes a relevance score for each document $D$ in the collection.
The scoring process takes into account both the query text and the intent type. More precisely, we use the
$Similarity(Q.text, D)$ and $TypeClassifier(Q.type, D)$ scoring functions to compute the relevance of the document to the query text and the
query intent type, respectively.

The final scoring function is a fusion between the two scoring functions:
\[
 FusedScore(Q, D) = Fusion\left(Similarity(Q.text, D), TypeClassifier(Q.type, D)\right)
\]
As stated in the introduction, the aim of the paper is to determine whether using the query intent type leads to an improvement
over the baseline (i.e., the similarity score). Considering $\mathcal{P}: f \rightarrow \mathbb{R}$ the performance function induced
by a scoring function $f$,
our goal is to maximize
\[
 \mathcal{P}(FusedScore(Q,D)) - \mathcal{P}(Similarity(Q.text,D))
\]


Let us now look at each of the three components in the formula individually.

\section{Text Similarity}
For computing the similarity between a query and a document, we use the well-known BM25 function. We therefore have:
\[
Similarity(Q.text, D) = BM25(Q.text,D),% = \displaystyle\sum_{term\in Q.text} \text{IDF}(term)\cdot\displaystyle\frac{freq(term,D)\cdot(k_1+1)}{freq(term,D)+k_1\cdot\left(1-b+b\cdot\frac{|D|}{avgDocLen}\right)}
\]
For each query, we also max-normalize the BM25 scores so that the top result has a score of 1.

\section{Intent Type Matching}
Intent type matching should tell us whether a particular document is useful for the intent type of the query 
(diagnosis, test or treatment). There are various approaches to do this, but in this paper we mainly look at
binary classifiers and word-counting for determining whether a document conforms to the intent type of a query.

For each intent type $T$ we construct a classifier and a word-counter in the following way:
\begin{itemize}
 \item the classifier gives a $ClassifierScore(T,D)$ which measures if document $D$ conforms to type $T$;
 \item the word counter gives a score which measures the frequency of certain keywords related to type $T$ in document $D$:
 
 \[WordCountScore(T,D) = \displaystyle\frac{\displaystyle\sum_{t\in \text{Keywords}(T)} \text{count}(t,D)}{\text{cutoff}(T)},\]
  where the cutoff is a positive number determined experimentally.
\end{itemize}


The final intent type matching score is the average of the two scores:
\[
 TypeClassifier(T, D) = \displaystyle\frac{ClassifierScore(T,D) +  WordCountScore(T,D)}{2}
\]
While the word-counter is straightforward, the classifier is a more complex component. As a result, we investigate both
classical approaches, as well as modern ones (deep-learning).

\subsection{Classical Classifiers}

We use a wide array of classifiers, such as Stochastic-Gradient-Descent-based, with several loss functions:
    \begin{itemize}  \setlength\itemsep{-0.5em}
    \item \emph{log} loss, which gives a Logistic Regression Classifier;
    \item \emph{hinge} and \emph{squared hinge} losses, which give a Support Vector Classifier;
    \item \emph{squared} and \emph{epsilon insensitive} losses;
    \end{itemize}
Other classifiers are: Naive Bayes (Multinomial and Bernoulli variations), Ridge, Perceptron, Passive-Agressive, Nearest Centroid
and Random Forests.

\subsection{Convolutional Neural Network Classifiers}
TODO

\section{Fusion of Text Similarity and Intent Type Matching}
Now that we have the similarity score and the intent type score, we need to fuse them to produce the final score.
We employ two main methods of fusion: unsupervised and supervised.

\subsection{Unsupervised Fusion}
There are many ways to fuse two scores, these are the ones we look at in this paper:

\begin{itemize}
 \item \emph{Linear Interpolation}, where the final score is given by: 
 \begin{equation} \label{interp-formula}
 Score(Q, D) = \lambda \cdot Similarity(Q_{\text{text}}, D) + (1-\lambda) \cdot TypeClassifier(Q_{\text{type}}, D)
 \end{equation}
where $\lambda\in [0,1]$ is determined experimentally;

\item \emph{Weighted Reciprocal Rank Fusion}, where the final score is given by:
 \begin{equation}\label{rrf-formula}
  Score(Q,D)\ =  \displaystyle\frac{\lambda}{k + r_1(D)} + \displaystyle\frac{1- \lambda}{k + r_2(D)}
 \end{equation}
  where $r_1(D)$ and $r_2(D)$ are the ranks of $D$ in the ordered list of scores induced by $Similarity$ and $TypeClassifier$ for query $Q$, respectively.
 
 \item \emph{Weighted Borda Fusion},  where the final score is given by:
 \begin{equation}\label{borda-formula}
    Score(Q,D)\ =  \displaystyle\frac{1}{\lambda \cdot r_1(D) + (1-\lambda)\cdot r_2(D)}
 \end{equation}
  
  \end{itemize}

\subsection{Supervised Fusion}
With supervised fusion, we use a subset of the queries as training data, and test on the remaining queries.
We try several Learning to Rank algorithms: \emph{MART}, \emph{RankNet}, \emph{RankBoost}, \emph{AdaRank},
\emph{Coordinate Ascent}, \emph{LambdaRank}, \emph{LambdaMart}, \emph{ListNet} and \emph{RandomForest}.
These algorithms take as input a list of features for each query-document pair and learn how to rank documents from the training
data.
The features we use are the two scores from before (similarity and type-match).


