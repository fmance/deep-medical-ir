\begin{abstract}
The amount of medical literature is continuously expanding,
making it increasingly demanding for physicians to locate medical information
relevant to their patients' cases. Clinical Decision Support (CDS) addresses part
of this problem by using artificial intelligence to help physicians with their clinical
decision-making needs.

In this paper, we focus on CDS scenarios where physicians have the medical case report of a patient
and need assistance with one of the following generic clinical tasks:
making a \emph{diagnosis}, performing medical \emph{tests}, or
determining the course of \emph{treatment}.
The goal is to build an information retrieval system that
retrieves bio-medical articles relevant to both the medical case report (the \emph{query text}),
and to the generic clinical question posed by the physician (the \emph{query
intent type} --- diagnosis, test or treatment).
% Our goal is to determine whether incorporating intent type along with textual relevance, into the retrieval process
% leads to an improvement in precision compared to only using textual relevance.

% Building information retrieval systems addressing this particular type of clinical need
% % containing both a textual component (the medical case report) and an intent type component (diagnosis, test or treatment),
% has been the goal of TREC's CDS track,
% where the retrieved biomedical articles must be relevant to both the medical case report, i.e., the \emph{query text},
% as well as to the generic clinical question posed by the physician, i.e., the \emph{query
% intent type} (diagnosis, test or treatment).

% The few participating systems that used both components of the queries for retrieval involved a machine-learning
% \emph{classifier} to categorize documents by their intent types,
% and a \emph{score fusion} method to combine the classification scores with the query text relevance scores.
% Due to the limited number of submissions allowed by TREC, however, these systems only tried one type of classifier and one type
% of fusion method, and did not investigate any further approaches.

We use \emph{machine-learning classifiers} and \emph{keyword counters} to compute the relevance of documents to the queries' intent types,
and \emph{score fusion} to combine the classification scores with textual relevance scores (computed by BM25).
We experiment with a wide range of classifiers
(Support Vector Machines, Logistic Regression, Ridge Classification, Perceptrons, Passive-Aggressive Classification,
Neural Networks, Multilayer Perceptrons),
as well as with several unsupervised and supervised fusion methods (Linear Combination, Reciprocal Rank Fusion,
Borda, Linear Regression, Ada Rank, Rank Net, Coordinate Ascent, MART, Lambda MART and Rank Boost).

We analyze and compare each of these
approaches in detail, by applying our retrieval system on the TREC CDS tracks from 2014 and 2015.
Our results show that incorporating intent type information into the retrieval process
(via classification and score fusion) leads
to solid improvements in Precision@10 compared to only using textual relevance:
up to 8\% absolute improvement for 2014 and up to 4\% absolute improvement for 2015.
% Our best method outperforms the best participating system
% from TREC 2014 in terms of Precision@10,
% and also the best system from TREC 2015 that used classifiers and fusion for incorporating intent types into the retrieval.

\end{abstract}
