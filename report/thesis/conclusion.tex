\chapter{Conclusion}

% Let us now summarize the main contribution of the paper to the Clinical Decision Support area.

In this paper, we focused on Clinical Decision Support systems addressing
a specific type of medical query, containing both a \emph{textual} component
(describing the patient's symptoms, clinical history, any test results)
as well as an \emph{intent type} component (the generic clinical question the physician is interested in
--- what the \emph{diagnosis} is, what \emph{tests} should be performed, how the patient should be \emph{treated}).

To incorporate intent types into the document retrieval process, we used
\emph{machine-learning classifiers} to categorize documents by their intent types,
and \emph{score fusion} methods to combine the classification scores with the query text relevance scores.

Our main contribution was investigating this approach in a much more comprehensive and systematic way
than previous research.
We examined a wide range of \emph{machine-learning classifiers}
% (Support Vector Machines, Logistic Regression, Ridge, Passive Aggressive, Naive Bayes, Perceptrons,
% Convolutional Neural Networks, Multilayer Perceptrons),
as well as several supervised and unsupervised \emph{score fusion} methods.
% : unsupervised
% (Linear Combination, Reciprocal Rank Fusion and Borda Fusion),
% and supervised (AdaRank, RankNet, Regression, Coordinate Ascent, MART, Lambda MART and RankBoost).
We applied our retrieval system on the TREC CDS tracks from 2014 and 2015 and analyzed
the performance and behavior of each of our models in detail.

The results clearly show that using intent types leads to consistent improvements in retrieval performance
compared to only using textual relevance.
The best results were obtained using SVM with Precison@K loss for classification,
and Linear Combination for score fusion.
For TREC CDS 2014, this yielded an 8\% absolute improvement in Precision@10 over the baseline of 31.67\%,
and for TREC CDS 2015, it yielded a 4\% absolute improvement in Precision@10 over the baseline of 38.67\%.

Several interesting directions for future work are:
1) using the TREC CDS 2016 data to train the supervised fusion methods and enhance system's robustness,
and 2) investigating other training sets for the
intent type classifiers, containing medical articles whose full text is available for download
(possibly from the Open Access subset of PubMed Central).

