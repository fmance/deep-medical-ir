\chapter{Related Work} \label{related}

Medical information retrieval is a highly active area of research that has extensively improved over time.
In this paper, we mainly focus on the Clinical Decision Support (CDS) area, which aims to help
physicians with their clinical information needs.

In the previous chapter, we defined our goal to be
an information retrieval system addressing a specific type of medical query,
containing both a \emph{textual} component (the patient record), as well as an \emph{intent} component
(the generic clinical question the physician is interested in
--- what the \emph{diagnosis} is, what \emph{tests} should be performed, how the patient should be \emph{treated}).

Since this type of medical questions was featured prominently in the TREC CDS tracks from 2014 and 2015,
we focus our attention on the work carried out by the participants in these tracks, as well as in
 their associated research.

We mentioned in the introductory chapter that, even though the retrieved medical documents should be
relevant to both the queries' text and to their intent types, the majority of participants
did not actually take the latter into account, focusing instead on adapting existing
IR techniques to the technical, specialized domain of medical queries. We list here some of the most
popular approaches that were investigated, although very briefly, since we want to concentrate more on the systems that
tried to incorporate the intent types into the retrieval process.

Most participants used existing IR systems such as
Lucene \cite{ucla, goodwin2014utd, jiang2016clinical, oh2014kisti, novasearch, CSEIITV, wei2014atigeo},
Terrier \cite{dinh2014crp, snomeddawit, bitem, waterloo, song2015ecnu, mutrec, sankhavara2014fusing, cuhk},
Indri \cite{ir.cs.sfsu, choi, duth, wei2014atigeo, wsuir},
Solr \cite{waterloo},
or Elasticsearch \cite{lamda2015},
together with a variety of ranking models, such as
BM25 \cite{dinh2014crp, snomeddawit, bitem, goodwin2014utd, novasearch, song2015ecnu},
TFIDF \cite{dinh2014crp, ucla, novasearch},
unigram language models \cite{ir.cs.sfsu, choi, ucla, oh2014kisti, novasearch},
BB2 \cite{song2015ecnu},
PL2 \cite{waterloo, song2015ecnu},
or InL2c1 \cite{snomeddawit}.

To improve retrieval performance, several techniques such as
query expansion \cite{mutrec, sankhavara2014fusing, cuhk, CSEIITV, ucla, udel, duth, ir.cs.sfsu, goodwin2014utd, oh2014kisti},
concept extraction \cite{ir.cs.sfsu, ucla, bitem, goodwin2014utd, oh2014kisti}
and pseudo-relevance feedback \cite{novasearch, waterloo, mutrec, oh2014kisti, choi, dinh2014crp}
were also used with varying degrees of success.
Some participants tried expanding the queries with medical concepts
from the UMLS, MeSH or MetaMap databases \cite{lamda2015, ucla, udel, bitem, novasearch},
or used negation detection to differentiate the negated terms in the query
\cite{wing2014query, wei2014atigeo, ucla, oh2014kisti}.
Others searched only certain sections of the documents, such as the abstracts, in order to make the search more focused
and remove potential ``noise'' from the documents' bodies \cite{ucla}.
Other successful methods were using Markov Random Fields \cite{wsuir},
\emph{n}-gram indexing \cite{xu2014hltcoe}
or experimenting with combinations of several ranking models,
such as BB2 and PL2 \cite{song2015ecnu}.

Let us now turn our attention to the systems that tried to use the query intent types to improve retrieval performance.
Most of them used the intent types either for query expansion or for document classification.

\section{Using Intent Types for Query Expansion}

The most popular approach was using the intent types for query expansion.
For example, Wan et al.\ \cite{cuhk} used WordNet to expand the query with noun synonyms
of \emph{diagnosis}, \emph{test} and \emph{treatment} such as \emph{designation} and \emph{identification},
but the approach proved detrimental overall.
% to the query, but unfortunately this approach proved detrimental, as their Precision@10
% (P@10) declined from 27\% to 23\%.

A similar idea was used by Singh et al.\ \cite{CSEIITV}, who expanded the queries with hand-crafted synonyms
for each of the three query types, but their results were generally poor.
% , with a P@10 of less than 6\%.

Garcia-Gathright et al.\ \cite{ucla} also expanded the query with synonyms of the intent types, such as \emph{test}, \emph{evaluate},
\emph{diagnose}, \emph{examination}, as well as with UMLS semantic types such as \emph{Diagnostic Procedure}
or \emph{Laboratory Procedure}. However, we cannot know what the contribution
of the query expansion is, since they do not provide a baseline without it.
% Their highest P@10 was around 25\%,

UMLS semantic types were also used by Wang et al.\ \cite{udel} who expanded the query
with types such as \emph{is\_finding\_of\_disease}, \emph{treated\_by} or
\emph{may\_be\_prevented\_by} together with terms extracted from the Cases Database to weigh certain keywords
based on the query intent type. The approach was successful, improving their overall retrieval precision.
% This improved their P@10 from 28.67\% to 31.67\%.

A similarly successful approach was taken by Gobeill et al.\ \cite{bitem}, who boosted documents based on the
frequencies of certain MeSH
concepts associated with the query intent type (such as \emph{MeSHtargetDiagnosis}).
% This improved their P@10 from 18.67\% to 20.33\%.

Drosatos et al.\ \cite{duth} added synonyms from the UMLS Metathesaurus for each intent type, for example
\emph{treat, treatment, treating, therapy, management}. This, however, decreased their overall precision.
% however their P@10 decreased by 1.33\% compared to the baseline.

Bhandari et al.\ \cite{ir.cs.sfsu} assigned a score to each document based on
its textual similarity to a list of several hundred test and treatment terms from a database\footnote{\url{rightdiagnosis.com}, \url{webmd.com}},
and used these scores to boost the retrieved test and treatment documents, but obtained poor results.
% with a P@10 of 18.67\%.

A similar approach was taken by Mourao et al.\ \cite{novasearch}, who selected MeSH
terms for each of the three intent types and assigned a weight to each
document based on how many of these terms it contained. Unfortunately, this approach proved detrimental to
the overall performance.
% This proved to be detrimental, as their P@10 decreased from 39\% to 36.67\%.
%
% We have now examined the previous research that used keywords related to the query intent types for improving
% retrieval. As we have seen, the approach met with varying degrees of success, depending on the actual
% strategy that was used. When not carefully designed, it backfired, such as in \cite{novasearch}, \cite{cuhk},
% \cite{CSEIITV}, \cite{ir.cs.sfsu}  or \cite{duth}. However, it was also successful for \cite{bitem} and \cite{udel}.

\section{Using Intent Types for Document Classification}\label{related-intent-type}
Let us now explore the strategies that used machine-learning classifiers to
incorporate intent types into the retrieval process.

Soldaini et al.\ \cite{soldani} used a supervised SVM classifier trained on four hundred medical articles
labeled by medical experts into the three intent types. The retrieved documents were classified
and grouped into clusters using the affinity propagation algorithm and ranked in the order of their relevance
to the query intent type. However, this approach was unsuccessful, yielding the lowest precision out of their five submissions.
% Unfortunately, this run had the lowest P@10, 18\%, out of their five submissions.

D’hondt et al.\ \cite{limsi2015} trained a classifier on ten thousand documents from PubMed Central
for each query intent type. They used the Winnow algorithm from the Linguistic Classification System
for classification and the weighted Borda Fusion method for combining the original retrieval scores
with the classifier scores, the latter receiving a weight of 0.2. However, it is
unclear how much the classifier itself helped, since they do not provide a baseline without it.
% Their best run had a P@10 of 37.67\%,

A similar approach was taken by Choi et al.\ \cite{choi},
who trained their classifier on the Clinical Hedges Database (CHD) \cite{chd}.
The database contains approximately 50,000 documents categorized
intro eight classes, including diagnosis and treatment.
It does not contain, however, any documents for the test type, so Choi et al.\ used the diagnosis documents instead,
since the two intent types are relatively similar.
They used Support Vector Machines with ROC-Area loss for classification,
and Borda Fusion for combining the baseline retrieval
scores with the classifier scores, as in their previous related research from \cite{choi-prev}.
This approach improved their baseline precision, however, they obtained better results when \emph{not} reranking the diagnosis queries.
% Their P@10 improvements range from 32\% to 33.67\% when using
% the short form of the queries (the summaries) and from 32\% to 33\% when using the long form of the queries
% (the descriptions).
% However, they have better results when \emph{not} reranking the diagnosis queries: 34.67\% and 36.33\%,
% which means that their diagnosis classifier did not work as expected and worsened the original, baseline, rankings.

You et al.\ \cite{FDUMedSearch} also used the Clinical Hedges Database for training a classifier, but only for the treatment type.
% Their best P@10 was 39.33\%, but
They do not disclose how their classifier and fusion method are implemented
neither do they provide a baseline without the classifier, meaning that we cannot determine its contribution.

\section{Conclusions}
Now that we have studied the previous research in
using intent types for Clinical Decision Support, we can draw two main conclusions:
\begin{enumerate}
 \item the classification and fusion approach yielded some promising results;
 \item however, it has not yet been sufficiently explored --- \cite{choi, FDUMedSearch, soldani, limsi2015}
 all used a single, fixed configuration for their classifier and fusion methods.
\end{enumerate}

In the introductory chapter, we stated that our over-arching goal was to
explore this idea in a more extensive and systematic way by investigating
many more classifiers and fusion methods than the previous participants did, and in the end,
be able to make a better recommendation based on our comprehensive and extensive study.

In order to do this, let us proceed to the first step --- formally defining our system.


