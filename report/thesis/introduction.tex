\chapter{Introduction}
The purpose of Clinical Decision Support is to help physicians with their clinical information and decision-making needs.
This being a very active area of research, the Text Retrieval Conference (TREC)
introduced in 2014 the \emph{Clinical Decision Support} (CDS) track,
whose goal is to simulate real-world case scenarios and to gather world-wide, state-of-the-art, research in the area.

The TREC CDS track focuses on situations where physicians have the \emph{case report} of a patient 
--- containing their medical history, current symptoms, test results and any other relevant information --- 
and need assistance with either one of these three generic
\emph{clinical questions}: 
\begin{itemize}[noitemsep,nolistsep]
 \item what is the patient's diagnosis ?
 \item what test should be performed ?
 \item how should the patient be treated ?\end{itemize}

The purpose of the track is to build an \emph{information retrieval} (IR) system that
retrieves \emph{medical articles} relevant to both the medical
case report as well as to the generic clinical question the physician is interested in.

In more formal, IR, terms, we are asked to retrieve \emph{documents} for \emph{queries} that have two components:
\begin{itemize}[noitemsep, nolistsep]
 \item a \emph{textual} component --- the medical case report;
 \item an \emph{intent type} component --- one of the three clinical questions above.
\end{itemize}

Both query components are taken into account when evaluating document \emph{relevance}.
The \emph{challenge} is, therefore, not only retrieving medical articles for queries written in a highly-specialized and technical
vocabulary, but also taking into account the \emph{intent} of the queries. 
In other words, it 
is not enough to only use the medical case report (query text)
for retrieval, as the documents might not conform the generic clinical question (query type).


However, most participants from the 2014 and 2015 editions of TREC CDS did not focus their efforts on using the intent types for retrieval,
and used only the query text instead. Most of their efforts were concentrated on making the query more meaningful by using techniques
such as query expansion, pseudo-relevance feedback, expansion using medical terms from UMLS, MeSH or MetaMap and tweaking
the scoring function of their IR systems.

A few participants did, nevertheless, try to incorporate the intent type in the retrieval process. Most of them
tried to expand the query using terms and synonyms of the intent types (i.e., diagnosis, test, treatment and their synonyms).
The best systems, however, used a machine-learning \emph{classifier} to categorize the retrieved documents according
to the three intent types and, together with the textual similarity score, use it to improve the overall retrieval performance.
The approach proved successful in some cases, leading to improvements compared to only using the textual similarity.

However, the full potential of the approach has not yet been explored enough,
since most participants have only ``scratched the surface'', that is, they only tried a single configuration of
the classifier and the fusion method, without delving any deeper into the realm of other possibilities.

In this paper, we pick up on the idea of incorporating the intent type relevance into the retrieval system,
but we explore a \emph{much wider array of strategies and algorithms} than in previous research. Specifically,
we use several varieties of machine-learning classifiers, ranging from classical ones, like Support Vector Machines or
Logistic Regression, to more modern, deep-learning ones, like Convolutional Neural Networks or Multilayer Perceptrons.
For the fusion of textual and intent type similarities,
we examine both unsupervised fusion methods, such as Weighted Linear Combination, Reciprocal Rank Fusion and Borda Fusion,
as well as supervised methods based
on Learning-to-Rank algorithms such as AdaRank, RankNet, Regression, Lambda MART and RankBoost.

Our goal is to design and implement a framework that allows us
to explore this idea as much as possible, try out all of the methods above, 
analyze their performance in detail,
determine which ones work best and compare them with existing research.
For each approach, we want to analyze its advantages, disadvantages and trade-offs,
and, in the end, be confident enough to be to make a recommendation on what we think is the best one.

The most significant advantage that we have compared to previous participants in the TREC CDS editions is 
that we have \emph{more data}. Previous participants from 2014 had no prior data, while the ones from 2015
only had the 2014 data. This limitation hampered their ability to properly train and evaluate their models, since they
had too little data to make a definitive judgement. 

We, however, have both the 2014 and 2015 data, which means that
we have more queries and more relevance judgements.
Even so, we still do not have as many as we would like in order to be able to give an unequivocal answer.
For example, we will find during the results' analysis
that some parameters that work very well for 2014 are bad for 2015 and vice-versa. Having two
years to evaluate our models allows us to choose configurations that work better \emph{on average}, but we
may still be a certain way off the optimal configuration.

Having data from two years also allows us to use one year for training and another one for testing.
This enables us to use supervised learning-to-rank algorithms,
something that none of the previous participants could have tried.

The main contribution of our work is, therefore, studying a very broad range of methods of incorporating
the intent type information into the retrival process, 
facilitated by us having more data than previous research to confidently evaluate the resulting models.

We evaluate our systems on both the 2014 and the 2015 queries; we use textual similarity
as a \emph{baseline} and then look at how much each of our approaches can potentially \emph{improve} upon this baseline.
We investigate what the best parameters are, how each method behaves for the three
intent types and what the improvements for each query are.

After investigating all of these approaches and performing an in-depth analysis of their results,
we are confident that we can make a recommendation.
Our best approaches improve the 2014 baseline by at least 8\%, yielding a score better than all the other systems from TREC 2014,
and the 2015 baseline by 3-4\%, yielding a score better than all the other systems from TREC 2015 that tried to incorporate the intent
type into the retrieval process.

Let us now briefly discuss the overall structure of the paper. In Section \ref{related} we go over the previous work
in the CDS domain, with a strong focus on the papers from the TREC CDS track. 
In Section \ref{meth} we give the formal definition of our retrieval system, discussing how we classify documents
and how we incorporate the intent type scores. 
In Section \ref{impl} we present the practical realization of the retrieval system, including any external libraries
and frameworks that we use.
In Section \ref{exp} we apply our system to the TREC 2014 and 2015 data, and evaluate each of our models.
We conclude by doing a comparison between our best methods and previous research.







